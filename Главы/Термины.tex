\chapter{Список терминов}

\parindent=0cm

Атрибут - специальный класс в C\#, позволяющий добавлять классам и их членам дополнительные метаданные

Вершина - позиция с дополнительной информацией, такой, как цвет, нормаль и UV-координата

Грань - замкнутое множество рёбер, чаще всего имеющее форму треугольника

Кватернион - гиперкомплексное число, образующее векторное пространство размерностью четыре над полем вещественных чисел и определяемое суммой $a + bi + cj + dk$, где $i$, $j$, $k$ - мнимые единицы, такие, что $i^{2} = j^{2} = k^{2} = ijk = -1$

Кривые Безье - тип параметрических кривых, описываемых с помощью опорных точек

Линейная интерполяция - функция, вычисляющая значение между $a$ и $b$ на основе приращения $t$

Материал - описание визуальных свойств модели, например, отражающей способности, используемых шейдеров и текстур

Префаб - специальный тип ассетов Unity, позволяющий хранить объект вместе со всеми компонентами и значениями свойств

Полигон - набор граней, лежащих в одной плоскости

Полигональная сетка - совокупность вершин, рёбер и граней, определяющих форму объекта

Процедурная генерация - программная генерация контента с использованием специальных алгоритмов

Ребро - соединение между двумя вершинами

Сериализация - сохранение структуры данных в виде последовательности битов

Текстура - плоское изображение, накладываемое на полигональную сетку

Трёхмерная модель - изображение объекта в трёхмерном пространстве

Шейдер - программа, испольняемая видеокартой

C\# - объектно-ориентированный язык программирования, используемый для написания скриптов и расширений для редактора в Unity

L-системы - вид формальных грамматик, основанный на переписывании строки

Low poly - минималистичный стиль графики, использующий сравнительно низкое число полигонов и раскраску вершин

Unity - кроссплатформенный игровой движок, поддерживающий более 25 платформ

UV-координаты - отображение координат вершин на координаты текстуры

\parindent=1.25cm
