\chapter{Введение}

Для создания контента в мультимедиа, включая графику, широко
применяется процедурная генерация, использующая специальные
алгоритмы. Процедурно сгенерированный контент может быть
доработан вручную перед включением в игру или же использован
непосредственно, генерируясь «на лету». Так или иначе, процедурная
генерация упрощает работу разработчикам и позволяет добиться
большего разнообразия контента. С её помощью можно генерировать
в том числе и правдоподобную графику, которая выглядит как 
сделанная человеком.

Одной из самых распространённых декораций в видеоиграх и фильмах
являются деревья. Деревья используются часто и в большом количестве, но при этом являются лишь второстепенной частью окружения, а следовательно, неудачно сгенерированное дерево практически не влияет на погружение. В силу этих причин имеет смысл генерировать деревья процедурно, чтобы
обеспечить вариативность, делая локации более правдоподобными и
упрощая ориентирование по ним. 

Ярким примером программного обеспечения для процедурной генерации растительности является система SpeedTree, выпущенная в 2002 году. С тех пор SpeedTree была использована для генерации деревьев в многочисленных играх крупнейших студий, например, Bethesda Softworks, Ubisoft и CD Projekt Red. SpeedTree широко используется в фильмах, начиная с <<Аватара>> Джеймса Кэмерона, вышедшего в 2009 году. Помимо этого, она используется в различных приложениях-симуляторах, к примеру, в симуляторе лесных пожаров, разработанном в Университете Центральной Флориды.

Для процедурной генерации моделей растительности в большинстве случаев применяются системы Линденмайера, или L-системы. L-система - это вид формальной грамматики, изначально предназначенный для моделирования процесса развития растения. Их рекурсивная природа приводит к самоподобию, позволяя описывать подобные фракталам органические формы. Таким образом можно получить сложные, детализированные и реалистичные модели.

Однако деревья, сгенерированные с помощью систем наподобие SpeedTree, подходят лишь для проектов с реалистичной и детализированной графикой. Независимые разработчики зачастую используют стилизованную графику в стиле low-poly - техники, в которой используется сравнительно небольшое число полигонов. Такой стиль обладает рядом преимуществ по сравнению с реалистичной графикой, наиболее важное из которых - низкий порог вхождения. Поскольку модели построены на графических примитивах, создавать их может даже программист без способностей к рисованию и моделированию. Кроме того, при небольшом количестве полигонов можно создавать достаточно красивые модели без наложения текстур, используя раскраску вершин.

В рамках данной работы был разработан плагин для движка Unity 3D, позволяющий генерировать стилизованные модели деревьев, подходящие для использования в качестве внутриигровых объектов. 
 
\newpage
