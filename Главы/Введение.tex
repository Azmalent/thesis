\chapter{Введение}

В мультимедиа и в частности в играх широко применяется процедурная генерация, использующая специальные алгоритмы для создания контента самостоятельно или при взаимодействии с разработчиками. Процедурная генерация может применяться для генерации уровней, объектов, персонажей, квестов, текстур, анимации и многого другого. Целью использования процедурной генерации может быть повышение реиграбельности и ускорение создания разнообразного контента.

Одним из первых примеров игр, применяющих процедурную генерацию контента, является Rogue. В этой ролевой игре 1980 года игрок брал на себя роль искателя приключений, исследующего процедурно сгенерированное подземелье. \cite{Rogue} Rogue породила целый жанр игр, названный в её честь - roguelike. В таких играх большая часть контента, включая уровни, предметы и противников, генерируется процедурно.

Процедурная генерация также применяется для генерации графики. В частности, возможно применять её для генерации внутриигровых объектов, чтобы упростить работу разработчикам и освободить время на работу над геймплеем и сценарием. 

Деревья являются одной из самых распространённых декораций в видеоиграх. Они используются часто и в большом количестве, но при этом являются лишь второстепенной частью окружения, а следовательно, неудачно сгенерированное дерево практически не влияет на погружение. В силу этих причин имеет смысл генерировать деревья процедурно, чтобы обеспечить вариативность, делая локации более правдоподобными и упрощая ориентирование по ним. 

Ярким примером программного обеспечения для процедурной генерации растительности является система SpeedTree, выпущенная в 2002 году. С тех пор SpeedTree была использована для генерации деревьев в многочисленных играх крупнейших студий, например, Bethesda Softworks, Ubisoft и CD Projekt Red. SpeedTree широко используется в фильмах, начиная с <<Аватара>> Джеймса Кэмерона, вышедшего в 2009 году. Помимо этого, она используется в различных приложениях-симуляторах, к примеру, в симуляторе лесных пожаров, разработанном в Университете Центральной Флориды.

Для процедурной генерации моделей растительности в большинстве случаев применяются системы Линденмайера, или L-системы. L-система - это вид формальной грамматики, изначально предназначенный для моделирования процесса развития растения. Их рекурсивная природа приводит к самоподобию, позволяя описывать подобные фракталам органические формы. Таким образом можно получить сложные, детализированные и реалистичные модели.

Однако деревья, сгенерированные с помощью систем наподобие SpeedTree, подходят лишь для проектов с реалистичной и детализированной графикой. Независимые разработчики зачастую используют стилизованную графику в стиле low poly - техники, в которой используется сравнительно небольшое число полигонов. Такой стиль обладает рядом преимуществ по сравнению с реалистичной графикой, наиболее важное из которых - низкий порог вхождения. Поскольку модели построены на графических примитивах, создавать их может даже программист без способностей к рисованию и моделированию. При небольшом количестве полигонов можно создавать достаточно красивые модели без использования текстур, используя раскраску вершин. По этим причинам low poly снижает стоимость разработки проекта, освобождая время и ресурсы. Кроме того, низкополигональные модели отличаются более высокой скоростью загрузки и сниженными требованиями к аппаратному обеспечению по сравнению с высокополигональными, что особенно важно на мобильных платформах. \cite{LowPolyIntro} 

В рамках данной работы будет разработан плагин для движка Unity 3D, позволяющий генерировать стилизованные низкополигональные модели деревьев, подходящие для использования в качестве внутриигровых объектов. Деревья можно будет генерировать как непосредственно в игре, так и заранее. Для генерации дерева необходимо добавить к игровому объекту специальный скрипт, позволяющий настроить параметры. Для создания предустановок можно использовать систему префабов Unity.
 
\newpage
